\begin{abstract}

Apresenta-se, neste relatório, o que foi desenvolvido até o presente momento do projeto de doutorado sobre aprendizado de máquina no reconhecimento de padrões litológicos. Primeiramente, é apresentado a motivação da obra. Posteriormente, é explicado o que vem a ser redes neuronais e, juntamente apresento trabalhos já publicados e aplicados na área da perfilagem de poços. Em seguia explico os princípios matemáticos envolvidos na rede escolhida para resolver o problema proposto. E, ao final do capítulo $1$, mostro o que vem a ser o aprendizado não-supervisionado. O capitulo $2$ esclarece o contexto geológico da área que virá a ser estudada, nas etapas posteriores do projeto. No capítulo $3$, é proposto o método que será utilizado, ao longo do projeto, bem como quais são os objetivos. O capítulo $4$ ilustra a natureza do dado de \textit{well logging} e apresenta um teste de hipóteses realizado, na rede neuronal. No capítulo $5$, são mostrados os resultados desse teste, para as etapas de treinamento e identificação da rede. Estes testes apontaram que o erro da rede relativo à etapa do treinamento foi de $4\%$. E a estabilização da rede se deu com $1000$ ciclos de treinamento e com custo computacional de $20$ segundos, na compilação do programa.  E, por conseguinte, no capítulo $6$, são apresentadas as conclusões dos testes. São publicados, no capítulo $7$,  o cronograma de atividades do projeto atualizado, seguido, no final deste relatório, pelas referências bibliográficas.      

\end{abstract}

