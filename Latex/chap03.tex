\chapter{Método Proposto e Objetivo}

A parte operacional deste programa se divide em duas etapas: 1- Geração de dados sintéticos, 2- Treinamento e 3- Identificação. Cada uma destas etapas será realizada por um programa computacional específico, este programas vão funcionar de forma independente.

O primeiro programa tem por objetivo gerar dados sintéticos que devem simular os resultados obtidos num levantamento de um perfil composto.

O programa da etapa de Treinamento será alimentado com dados de perfilagens cujas respectivas fácies litológicas são conhecidas (inicialmente serão usados dados sintéticos e posteriormente dados reais). Este programa vai gerar um arquivo com os dados do treinamento, este arquivo será usado pelo programa de Operação. Esta é a fase de aprendizagem da rede.

O programa da etapa de identificação vai fazer a classificação, de forma autônoma, das facies litológicas em poços a partir de dados de perfilagem em poços nos quais a litologia é desconhecida. A aprendizagem da rede deve ocorrer de forma continuada, quanto mais informação temos sobre situações nas quais a litologia é conhecida mais bem preparada estará a rede em termos de aprendizagem. Este conceito de aprendizagem é acumulativo e isso ocorrerá através da atualização do arquivo com os dados de treinamento.

Durante a elaboração dos programas será necessário testar a usa eficiência. Estes testes serão realizados através de dados sintéticos que serão gerados por um terceiro programa, gerador de dados sintéticos. Este programa será alimentados com informações da literatura. Após os testes com dados sintéticos a metodologia será validada com dados reais, posteriormente depois de cumpridas todas estas etapas, a metodologia estará pronta para ser utilizada em situações reais.

É importante salientar que neste método o conhecimento do funcional geofísico que rege a relação entre a litologia e a propriedades físicas das rochas não é necessário durante o processo. O conceito de inteligência artificial que será utilizado prescinde do funcional geofísico, o aprendizado é feito através da identificação de padrões recorrentes.

O ponto positivo desta metodologia é prescindir do funcional geofísico, que por vezes é desconhecido ou de alta complexidade o que exige uma modelagem matemática trabalhosa.

O ponto negativo é a necessidade de se ter muitos dados já analisados em situações conhecidas e variadas para a realização da etapa de treinamento. A etapa de treinamento tem um custo computacional alto.

\section{Objetivo}
O principal objetivo deste projeto é desenvolver um programa computacional do tipo “ machine learning ”, que será implementado na forma de uma Rede Neural Artificial (RNA) dentro do contexto da inteligência artificial. Este programa deve ter a capacidade de identificar, de forma autônoma, fácies litológicas a partir de dados de perfilagem de poços sem a necessidade do uso de um funcional geofísico.

É importante salientar que a metodologia que será desenvolvida neste projeto tem aplicação direta tanto na indústria de exploração mineral, quanto na de água, e na de petróleo e gás.
