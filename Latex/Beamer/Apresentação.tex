%\documentclass[aspectratio=169]{beamer} %para apresentações em widescreen
\documentclass[10pt]{beamer} %para apresentação normal
\logo{\includegraphics[height=1cm]{Imagens/logon.jpg}\vspace{220pt}}
\newcommand{\nologo}{\setbeamertemplate{logo}{}} % command to set the logo to nothing
\usetheme{default}
\usefonttheme{serif}
\usecolortheme{default}
%%%%%%%%%%%%%%%%%%%%PACOTES EXTRAS%%%%%%%%%%%%%%%%%%%
\usepackage[utf8]{inputenc}
\usepackage[T1]{fontenc}
\usepackage[portuguese, english]{babel}
\usepackage[round]{natbib}
\usepackage{hyperref} 
\usepackage{smartdiagram}
\usepackage{graphicx} % Required for including images
\usepackage{graphics}
\graphicspath{{figures/}} % Location of the graphics files
\usepackage{booktabs} % Top and bottom rules for table
\usepackage[font=small,labelfont=bf]{caption} % Required for specifying captions to tables and figures
\usepackage{amsfonts, amsmath, amsthm, amssymb} % For math fonts, symbols and environments
\usepackage{wrapfig} % Allows wrapping text around tables and figures
\usepackage{ucs}
\usepackage{amsmath}
\usepackage{amsfonts}
\usepackage{amssymb}
\usepackage{amsthm}
\usepackage{times}
\usepackage{makeidx}
\usepackage{lipsum} % Required to insert dummy text. To be removed otherwise
\usepackage{epstopdf}%adiciona imagens em formato eps no pdf.
\usepackage{subfigure}%cria ambientes de multifiguras
\usepackage{float}%coloca as figuras exatamente aonde você quer
%package[monochrome]{xcolor}%imprime o arquivo final em preto e branco
%\usepackage[left=2cm,right=2cm,top=2cm,bottom=2cm]{geometry}
\usepackage{lipsum} % Required to insert dummy text. To be removed otherwise
%\usepackage{multicol, blindtext}%cria figura na página inteira
\hypersetup{colorlinks,breaklinks=true,urlcolor=color2,citecolor=color1,linkcolor=color1,bookmarksopen=false,pdftitle={Title},pdfauthor={Author}}%Comando adaptado para o texmaker do ubuntu 12.4 LTS
\definecolor{color1}{RGB}{0,0,90} % Color of the article title and sections
\definecolor{color2}{RGB}{0,20,20} % Color of the boxes behind the abstract and headings
\usepackage{tikz}%pacote para fazer fluxogramas
\usetikzlibrary{calc,trees,positioning,arrows,chains,shapes.geometric,decorations.pathreplacing,decorations.pathmorphing,shapes,matrix,shapes.symbols}
\tikzset{
			>=stealth',
			punktchain/.style={
				rectangle, 
				rounded corners, 
				% fill=black!10,
				draw=black, very thick,
				text width=10em, 
				minimum height=1em, 
				text centered, 
				on chain},
			line/.style={draw, thick, <-},
			element/.style={
				tape,
				top color=white,
				bottom color=blue!50!black!60!,
				minimum width=8em,
				draw=blue!40!black!90, very thick,
				text width=1em, 
				minimum height=3.5em, 
				text centered, 
				on chain},
			every join/.style={->, thick,shorten >=0.1pt},
			decoration={brace},
			tuborg/.style={decorate},
			tubnode/.style={midway, right=0.1pt},
}
\usepackage{verbatim}%


%%%%%%%%%%%%%%%%%%%%%%%%%%%%%%%%%%%%%%%%%%%%%%%%%%%%%%%%%%%%%%%%%%%%%%%%%%%%%%%
%------------------------------FINAL DO PREÂMBULO------------------------------
%%%%%%%%%%%%%%%%%%%%%%%%%%%%%%%%%%%%%%%%%%%%%%%%%%%%%%%%%%%%%%%%%%%%%%%%%%%%%%%


\author[Carreira,V.R.]{Autor: Victor Ribeiro Carreira \\ Orientador: Cosme Ferreira Ponte Neto}
\title{Inteligência Artificial Aplicada  ao  Reconhecimento de Padrões Litológicos.}
%\subtitle{}
\institute{Pós-Graduação em Geofísica\\ Projeto de Doutorado}
\date{Fevereiro de 2017}
\subject{Apresentação do Projeto de Doutorado}
%\setbeamertemplate{footline}[frame number]
%\setbeamercovered{transparent}
%\setbeamertemplate{navigation symbols}{}
% Tela cheia
\hypersetup{pdfpagemode=FullScreen}

\usepackage{ragged2e}
\justifying



%----------------------------------------------------------------------------------------
%	SEPARAÇÃO DE SÍLABAS
%----------------------------------------------------------------------------------------

\hyphenation{co-o-pe-ra-ção} 
\hyphenation{a-com-pa-nha-do} 
\hyphenation{nor-ma-li-za-ção} 
\hyphenation{nor-ma-li-za-do}
%-------------------------------------------------------------------------------------------

\begin{document}

\bgroup
\makeatletter
\setbeamertemplate{footline}
\makeatother
{\nologo
\begin{frame}
%\titlepage
\begin{figure}
\centering
\includegraphics[scale=0.4]{Imagens/logonvertical.jpg} 
\end{figure}
\end{frame}
}
\maketitle
\egroup 
\addtobeamertemplate{navigation symbols}{}{\hskip6pt\raisebox{2pt}{\color{blue}\insertframenumber}}
\setcounter{framenumber}{0}
\AtBeginSection[]
{ \begin{frame}
\centering
\frametitle{Sumário}

\tableofcontents[currentsection,currentsubsection]% apresenta o sumário antes de cada seção

\end{frame} }


%%%%%%%%%%%%%%%%%%%%%%%%%%%%%%%%%%%%%%%%%%%%%%%%%%%%%%%%%%%%%%%%%%%%%%%%%%%%
%-----------------------------INTRODUÇÃO---------------------------------
%%%%%%%%%%%%%%%%%%%%%%%%%%%%%%%%%%%%%%%%%%%%%%%%%%%%%%%%%%%%%%%%%%%%%%%%%%%%

\section{Introdução}

\subsection{O que vem a ser uma rede neuronal artificial?}
\begin{frame}
	\frametitle{O que vem a ser uma rede neuronal artificial?}
				\color{blue} É um sistema artificial que visa simular numericamente os processos do relizados pelo sistema nervoso central. 
				
				\begin{itemize}
					\item Disponta como uma alternativa ao paradigma computacional de Von Neumann (instruções de programação sequencial);
					\pause
					\item Inspirada na neurociência, contudo não é realística em detalhe;
					\pause
					\item Métodos são derivados da Estatística Física;
					\pause
					\item Aplicações em ciência da computação, engenharia, geologia e geofísica. 
				\end{itemize}
\end{frame}

\begin{frame}
	\frametitle{O cérebro humano contém cerca de \\ aproximadamente $10^{11}$   de vários tipos}
	\begin{figure}
		\centering
		\includegraphics[scale=0.2]{Imagens/neuronio.png} 
		\caption{Ilustração esquemática de um neurônio.}
	\end{figure}
\end{frame}

\begin{frame}
	\frametitle{O cérebro humano contém cerca de \\ aproximadamente $10^{11}$   de vários tipos}
	\begin{figure}
		\centering
		\includegraphics[scale=0.2]{Imagens/dendrito.png} 
		\caption{Ilustração esquemática de um neurônio.}
	\end{figure}
\end{frame}

\begin{frame}
	\frametitle{O cérebro humano contém cerca de \\ aproximadamente $10^{11}$   de vários tipos}
	\begin{figure}
		\centering
		\includegraphics[scale=0.2]{Imagens/nucleo.png} 
		\caption{Ilustração esquemática de um neurônio.}
	\end{figure}
\end{frame}

\begin{frame}
	\frametitle{O cérebro humano contém cerca de \\ aproximadamente $10^{11}$   de vários tipos}
	\begin{figure}
		\centering
		\includegraphics[scale=0.2]{Imagens/soma.png} 
		\caption{Ilustração esquemática de um neurônio.}
	\end{figure}
\end{frame}


\begin{frame}
	\frametitle{O cérebro humano contém cerca de \\ aproximadamente $10^{11}$   de vários tipos}
	\begin{figure}
		\centering
		\includegraphics[scale=0.2]{Imagens/axonio.png} 
		\caption{Ilustração esquemática de um neurônio.}
	\end{figure}
\end{frame}

\begin{frame}
	\frametitle{O cérebro humano contém cerca de \\ aproximadamente $10^{11}$   de vários tipos}
	\begin{figure}
		\centering
		\includegraphics[scale=0.2]{Imagens/cone.png} 
		\caption{Ilustração esquemática de um neurônio.}
	\end{figure}
\end{frame}

\begin{frame}
	\frametitle{O cérebro humano contém cerca de \\ aproximadamente $10^{11}$   de vários tipos}
	\begin{figure}
		\centering
		\includegraphics[scale=0.2]{Imagens/terminal.png} 
		\caption{Ilustração esquemática de um neurônio.}
	\end{figure}
\end{frame}

\subsection{Um breve histórico}

\begin{frame}
	\frametitle{Um breve histórico}
\end{frame}

\subsection{Estado da arte}

\begin{frame}
	\frametitle{Estado da arte}
\end{frame}

\subsection{A rede de Kohonen}

\begin{frame}
	\frametitle{A rede de Kohonen}
	\includegraphics[scale=0.5]{Imagens/IntroKoho1.png} 
	
\end{frame}


\begin{frame}
	\frametitle{A rede de Kohonen}
	\includegraphics[scale=0.5]{Imagens/IntroKoho2.png} 
	
\end{frame}


\begin{frame}
	\frametitle{A rede de Kohonen}
	\includegraphics[scale=0.5]{Imagens/IntroKoho3.png} 
	
\end{frame}

\begin{frame}
	\frametitle{A rede de Kohonen}
	\includegraphics[scale=0.5]{Imagens/IntroKoho4.png} 
	
\end{frame}


\subsection{Treinamento não-supervisionado}


%%%%%%%%%%%%%%%%%%%%%%%%%%%%%%%%%%%%%%%%%%%%%%%%%%%%%%%%%%%%%%%%%%%%%%%%%%%%
%---------------------------------CONTEXTO GEOLÓGICO------------------------------------
%%%%%%%%%%%%%%%%%%%%%%%%%%%%%%%%%%%%%%%%%%%%%%%%%%%%%%%%%%%%%%%%%%%%%%%%%%%%

\section{Contexto Geológico}

\begin{frame}
	\frametitle{Contexto Geológico}
	\begin{figure}[H]
		\centering
			\includegraphics[scale=0.3]{Imagens/BaciaParana.jpg}
		\caption{Mapa de localização da Bacia do Paraná. }
		\label{mapa geologico}
	\end{figure}
\end{frame}



%%%%%%%%%%%%%%%%%%%%%%%%%%%%%%%%%%%%%%%%%%%%%%%%%%%%%%%%%%%%%%%%%%%%%%%%%%%%
%--------------------------METODOLOGIA------------------------------
%%%%%%%%%%%%%%%%%%%%%%%%%%%%%%%%%%%%%%%%%%%%%%%%%%%%%%%%%%%%%%%%%%%%%%%%%%%%

\section{Metodologia}

\begin{frame}
	\frametitle{Criação dos dados sintéticos}
\begin{footnotesize}
	\begin{figure}[H]
		\centering

		\begin{tikzpicture}
		[node distance=.5cm,
		start chain=going below,]
		\node[punktchain, join]  {Modelo de Bacia};
		\node[punktchain, join]   {Dados da Literatura};
		\node[punktchain, join]   {Estimar erros das propriedades físicas para dada litologia};
		\node[punktchain, join] {Definir taxa de amostragem};
		\node[punktchain, join, ] {Gerar dados sintéticos contaminados com o erro gaussiano estimado};
		
		\end{tikzpicture}
%		\caption{Fluxograma do programa de geração dos dados sintéticos}
	\end{figure}
	
\end{footnotesize}	
\end{frame}


\begin{frame}
	\frametitle{Programa de treinamento da rede}
	\begin{scriptsize}
		

	\begin{figure}[H]
		\centering
	
		\begin{tikzpicture}
		[node distance=.3cm,
		start chain=going below,]
		\node[punktchain, join]  {Início};
		\node[punktchain, join] (L1)  {Entrar com os dados do treinamento};
		\node[punktchain, join]   {Atualizar o peso do neurônio vencedor};
		\node[punktchain, join] (L2) {Fazer alteração nas vizinhanças do neurônio vencedor. Varre os dados de forma sequencial};
		\node[punktchain, join, ] {Avaliar o nível de treinamento da rede};
		\node[punktchain, join, ] {\color{red}Caso não esteja satisfatório volta-se para o Início e repete o processo de forma acumulativa};
		\node[punktchain, join, ] {\color{blue}Caso esteja satisfatório é o decretado o final do treinamento};	
		\end{tikzpicture}
	\end{figure}
\end{scriptsize}
\end{frame}

\begin{frame}
	\frametitle{Programa de identificação da rede}
	\begin{small}
		
		
		\begin{figure}[H]
			\centering
			
	\begin{tikzpicture}
	[node distance=.8cm,
	start chain=going below,]
	\node[punktchain, join]  {Entrada do dado não classificado};
	\node[punktchain, join] (L1)  {Determinação da distância entre os dados de cada neurônio };
	\node[punktchain, join]   {Armazena as distâncias};
	\node[punktchain, join] (L2) {Seleciona a menor distância};
	\node[punktchain, join, ] {Classifica o dado};	
	\end{tikzpicture}
		\end{figure}
	\end{small}
\end{frame}





\subsection{Objetivo}

%%%%%%%%%%%%%%%%%%%%%%%%%%%%%%%%%%%%%%%%%%%%%%%%%%%%%%%%%%%%%%%%%%%%%%%%%%%%
%---------------------------NATUREZA DO DADO--------------------------
%%%%%%%%%%%%%%%%%%%%%%%%%%%%%%%%%%%%%%%%%%%%%%%%%%%%%%%%%%%%%%%%%%%%%%%%%%%%

\section{Natureza do dado}

\subsection{Modelo proposto para gerar os dados sintéticos}


\begin{frame}
	\frametitle{Modelo proposto}
 \begin{scriptsize}
	\begin{table}[H]
		\centering
		\caption{Compilação de propriedades físicas usdas para inferência de litologia \citep{Telford_1993}.}
		\label{rock-properties1}
			\begin{tabular}{@{}llllllllll@{}}
				\toprule
				Rocha         & Densidade ($g/cm^{3}$) & Raios-Gama ($Ci/g$)& Potencial-Espontâneo ($mV$)&   \\ \midrule
				Conglomerado &     $2,50$  &       ---        &    ---        &      \\
				Arenito  &    $2,35$      &       $2,00\leftrightarrow4,00$       &     ---       &      \\
				Folhelho &   $2,40$       &      ---        &      ---      &    \\
				Argilito &     $2,55$   &          ---     &       ---     &     \\
				Siltito  &      $2,21$    &          ---     &       ---     &   \\
				Dolomita &     $2,70$    &        $8,00$       &   ---         &       \\
				Marga  &    $2,50$     &         ---      &    ---        &     \\
				Basalto  &     $2,99$    &          $0,50$     &    ---       &      \\
				Diabásio &    $2,90$    &         ---      &       ---     &     \\
				Lava &     $2,61$    &      $0,33$         &      ---      &      \\
				Granito &    $2,64$      &       $0,70\leftrightarrow4,80$        &      ---      &      \\
				Gabro &    $3,03$     &       ---        &     ---       &       \\
				Peridotito &   $3,15$    &      ---         &     ---       &      \\
				Quartzito &    $2,60$    &        $5,00$      &     ---       &    \\
				Xisto &   $2,64$    &         ---      &      ---      &    \\
				Gnaisse &    $2,80$     &      ---         &    ---        &        \\
				Serpentinito &    $2,78$     &   ---            &   ---     &        \\
				Anfibolito &  $2,96$       &          ---     &       ---     &        \\
				Eclogito &  $3,37$    &       ---        &      ---      &    \\
				Mármore &   $2,75$       &      ---         &     ---       &      \\ \bottomrule
			\end{tabular}
	\end{table}
\end{scriptsize}
\end{frame}

\begin{frame}
	\frametitle{Modelo proposto}
	\begin{scriptsize}
		\begin{table}[H]
			\centering
			\caption{Compilação de propriedades físcas usadas na inferência de porosidade, permeabilidade. \cite{Telford_1993}.}
			\label{rock-properties2}
			\begin{tabular}{@{}llllllllll@{}}
				\toprule
				Rocha   & Resistividade ($\Omega/m$) &  Neutrão ($API$) & Velocidade ($km/s$)  &    \\ \midrule
				Conglomerado &    $2\times10^{3}\leftrightarrow10^{4}$       &    ---           &     $1,80\leftrightarrow4,90$       &     \\
				Arenito  &    $1\leftrightarrow6,4\times10^{8}$       &      ---         &     $4,00\leftrightarrow4,30$       &   \\
				Folhelho &     $50\leftrightarrow10^{7}$      &      ---         &      $2,15\leftrightarrow3,30$      &   \\
				Argilito &     $10\leftrightarrow8\times10^{2}$      &       ---        &     ---       &      \\
				Siltito  &      $1\leftrightarrow100$     &      ---         &         $4,00\leftrightarrow6,20$    &         \\
				Dolomita &   $3,5\times10^{2}\leftrightarrow5\times10^{3}$        &    ---           &      $5,70\leftrightarrow6,00$      &      \\
				Marga  &     $3\leftrightarrow70$      &     ---          &     ---       &     \\
				Basalto  &     $10\leftrightarrow1,3\times10^{7}$      &     ---          &     $ 5,00\leftrightarrow5.80$         &     \\
				Diabásio &  $20\leftrightarrow5\times10^{7}$         &      ---         &     ---       &  \\
				Granito Porfirítico (seco) &     $1,3\times10^{6}$     &       ---        &     $5,80$       &    \\
				Granito Porfirítico (úmido) &  $4,5\times10^{3}$          &      ---         &     $ 5,00\leftrightarrow5.60$         &      \\
				Gabro &   $10^{3}\leftrightarrow10^{6}$       &      ---         &      $ 5,00\leftrightarrow5.80$        &     \\
				Peridotito (seco) &   $6,5\times10^{3}$        &    ---           &       ---     &    \\
				Peridotito (úmido) &    $3\times10^{3}$       &      ---         &      ---      &  \\
				Xisto &    $20\leftrightarrow10^{4}$       &        ---       &       ---     &   \\
				Gnaisse (seco) & $3\times10^{6}$          &         ---      &     ---       &   \\
				Gnaisse (úmido) &   $6,8\times10^{4}$        &       ---        &      ---      &   \\
				Tufa (seca) &      $2\times10^{3}$     &      ---         &     $1,80\leftrightarrow3,50$       &     \\
				Tufa (úmida) &     $10^{5}$      &     ---          &     ---       &      \\
				Mármore &  $10^{2}\leftrightarrow2,5\times10^{8}$         &       ---        &      ---      &    \\ \bottomrule
			\end{tabular}
		\end{table}
	\end{scriptsize}
\end{frame}



\begin{frame}
	\frametitle{Modelo proposto}
	\begin{figure}[H]
		\centering
			\includegraphics[scale=0.35]{Imagens/Modelo.png}
		\caption{Modelo Simplificado baseado em \cite{Sal2008}.}
		\label{modelo}
	\end{figure}
\end{frame}

\begin{frame}
	\frametitle{Modelo proposto}
	\begin{figure}[H]
		\centering
			\includegraphics[scale=0.37]{Imagens/PocoT1.png}
		\caption{Dado de perfilagem sintético, T1. }
		\label{T1}
	\end{figure}
\end{frame}


\begin{frame}
	\frametitle{Modelo proposto}
	\begin{figure}[H]
		\centering
			\includegraphics[scale=0.37]{Imagens/PocoC1.png}
		\caption{Dado de perfilagem sintético, C1.}
		\label{C1}
	\end{figure}
\end{frame}

\begin{frame}
	\frametitle{Modelo proposto}
	\begin{figure}[H]
		\centering
			\includegraphics[scale=0.37]{Imagens/PocoC2.png}
		\caption{Dado de perfilagem sintético, C2.}
		\label{C2}
	\end{figure}
\end{frame}


\begin{frame}
	\frametitle{Clusterização}
	\begin{figure}[H]
		\centering
			\includegraphics[scale=0.3]{Imagens/cluterpocoT1.png}
		\caption{Agrupamento de dados do poço T1.}
		\label{clusterT1}
	\end{figure} 
\end{frame}

\begin{frame}
	\frametitle{Clusterização}
	\begin{figure}[H]
		\centering
			\includegraphics[scale=0.3]{Imagens/cluterpocoC1.png}
		\caption{Agrupamento de dados do poço C1.}
		\label{clusterC1}
	\end{figure} 
\end{frame}

\begin{frame}
	\frametitle{Clusterização}
	\begin{figure}[H]
		\centering
			\includegraphics[scale=0.3]{Imagens/cluterpocoC2.png}
		\caption{Agrupamento de dados do poço C2.}
		\label{clusterC2}
	\end{figure} 
\end{frame}



\subsection{Dado Real}

\begin{frame}
	\frametitle{Localização dos poços}
	\begin{figure}[H]
		\centering
			\includegraphics[scale=0.25]{Imagens/Pocos.jpg}
		\caption{Localização dos poços de trabalho.}
		\label{real}
	\end{figure}
\end{frame}


%%%%%%%%%%%%%%%%%%%%%%%%%%%%%%%%%%%%%%%%%%%%%%%%%%%%%%%%%%%%%%%%%%%%%%%%%%%%
%-------------------------RESULTADOS E DISCUSSÕES---------------------------------------
%%%%%%%%%%%%%%%%%%%%%%%%%%%%%%%%%%%%%%%%%%%%%%%%%%%%%%%%%%%%%%%%%%%%%%%%%%%%

\section{Resultados e Discussões}

\subsection{Treinamento}

\begin{frame}
	\frametitle{Treinamento}
	\begin{figure}
		\centering
		\includegraphics[width=7.0cm]{Imagens/SOM1_2d.pdf}
		\caption{Mapa auto-organizado (a) no primeiro ciclo de treinamento.}
	\end{figure}
\end{frame}

\begin{frame}
	\frametitle{Treinamento}
	\begin{figure}
		\centering
		\includegraphics[width=7.0cm]{Imagens/SOM5_2d.pdf}
		\caption{Mapa auto-organizado (b) no quinto ciclo de treinamento.}
	\end{figure}
\end{frame}

\begin{frame}
	\frametitle{Treinamento}
	\begin{figure}
		\centering
		\includegraphics[width=7.0cm]{Imagens/SOM100_2d.pdf}
		\caption{Mapa auto-organizado (c) no centésimo ciclo de treinamento.}
	\end{figure}
\end{frame}

\begin{frame}
	\frametitle{Treinamento}
	\begin{figure}
		\centering
		\includegraphics[width=7.0cm]{Imagens/SOM1000_2d.pdf}
		\caption{Mapa auto-organizado (d) no milésimo ciclo de treinamento.}
		\label{SOMd}
	\end{figure}
\end{frame}

\begin{frame}
	\frametitle{Treinamento}
	\begin{figure}[H]
		\centering
			\includegraphics[scale=0.23]{Imagens/conv070917.png}
		\caption{Teste de convergência da rede.}
		\label{convergencia}
	\end{figure} 
\end{frame}

\subsection{Identificação}

\begin{frame}
	\frametitle{Identificação}
\begin{figure}[H]
	\centering
		\includegraphics[scale=0.45]{Imagens/IDC1.png}
	\caption{Dado de saída da rede para o poço de classificação C1.}
	\label{Class C1}
\end{figure} 
\end{frame}

\begin{frame}
	\frametitle{Identificação}
	\begin{figure}[H]
		\centering
			\includegraphics[scale=0.45]{Imagens/IDC2.png}
		\caption{Dado de saída da rede para o poço de classificação C2.}
		\label{Class C2}
	\end{figure} 
\end{frame}
	


%%%%%%%%%%%%%%%%%%%%%%%%%%%%%%%%%%%%%%%%%%%%%%%%%%%%%%%%%%%%%%%%%%%%%%%%%%%%
%-------------------------CONCLUSÕES---------------------------------------
%%%%%%%%%%%%%%%%%%%%%%%%%%%%%%%%%%%%%%%%%%%%%%%%%%%%%%%%%%%%%%%%%%%%%%%%%%%%

\section{Conclusões}

\begin{frame}
	\frametitle{Conclusões}
\begin{small}
	

	\begin{itemize}
		\item O teste de convergência da rede, Fig \ref{convergencia}, indicou que o número de erros não diminui após o milésimo ciclo de treinamento;
		\pause
		\item O resultado da Fig. \ref{SOMd} (d)  é mapa auto-organizado usado na identificação da rede cuja maior área de especialização está relacionada com o padrão sino;
		\pause
		\item As propriedades físicas de densidade, velocidade e raio-gama tem uma importância relativa maior, na classificação das litologias pela rede dos poços  C$1$ e C$2$ (diagramas de velocidades por densidade e o de velocidade por raio-gama), Fig. \ref{clusterT1}, Fig. \ref{clusterC1} e Fig. \ref{clusterC2};
		\pause
		\item A saída da rede aponta que o maior número de casos dos erros ocorreram em uma única classe de rocha, a do embasamento;
		\pause
		\item O menor número de erros relativos encontrados, no poço C$2$, Fig. \ref{Class C2}, deve-se a escolha da alocação do furo, no perfil e do po ço utilizado no treinamento. O poço C$2$ localiza-se em um baixo estrutural, atingindo menos de $1$km do embasamento. Entretanto, o poço C$1$, Fig. \ref{Class C1}, encontra-se em um alto estrutural, divergindo do poço C$2$ e produzindo, consequentemente, os maiores erros relativos encontrados. 
	\end{itemize}
\end{small}	
\end{frame}




%%%%%%%%%%%%%%%%%%%%%%%%%%%%%%%%%%%%%%%%%%%%%%%%%%%%%%%%%%%%%%%%%%%%%%%%%%%%
%-------------------------CRONOGRAMA---------------------------------------
%%%%%%%%%%%%%%%%%%%%%%%%%%%%%%%%%%%%%%%%%%%%%%%%%%%%%%%%%%%%%%%%%%%%%%%%%%%%

\section{Cronograma}		
\begin{frame}
\begin{table}[H]

	\flushleft
	
	% definindo o tamanho da fonte para small
	% outros possíveis tamanhos: footnotesize, scriptsize
	\begin{footnotesize}
		
		% redefinindo o espaçamento das colunas
		\setlength{\tabcolsep}{1pt}
		
		% \cline é semelhante ao \hline, porém é possível indicar as colunas que terão essa a linha horizontal
		% \multicolumn{10}{c|}{Meses} indica que dez colunas serão mescladas e a palavra Meses estará centralizada dentro delas.
		\rotatebox{0}{
			\begin{tabular}{|c|c|c|c|c|c|c|c|c|c|c|c|c|c|c|c|c|c|c|c|c|c|c|c|c|}\hline
				& \multicolumn{24}{c|}{Meses}\\ \cline{2-25}
				\raisebox{1.5ex}{Etapa} & 01 & 02 & 03 & 04 & 05 & 06 & 07 & 08 & {\color{red}09} & 10 & 11 & 12 & 13 & 14 & 15 & 16 & 17 & 18 & 19 & 20 & 21 & 22 & 23 & 24 \\ \hline
				
				Pesquisa na Literatura & X & X & X & X & X & X & X & X & {\color{red}X} & X & X & X & X & X & X & X & X & X & X & X & X & X & X & X\\ \hline
				Disciplinas & & & X & X & X & X & X & X &  {\color{red}X} & X & X & X & & & X & X & X & X & X & X & X & X & X & X \\ \hline
				Formulação da Rede & & & & & & & & X &  {\color{red}X} & X & X & X & X & X & X & X & & & & & & & & \\ \hline
				Treino & & & & & & & & & & & & & X & X & X & X & X & X & X & X & X & X & X & X \\ \hline
				Resultado & & & & & & & & & & & & & & & & & & & & X & X & X & X & X \\ \hline
				Artigo 1 & & & & & & & & & & & & & & & & & & & & & & & X & X \\ \hline
				Artigo 2 & & & & & & & & & & & & & & & & & & & & & & & & \\ \hline
				Tese & & & & & & & & & & & & & & & & & & & & & & & & \\ \hline
			\end{tabular}
		}
	\end{footnotesize}
	\caption{Cronograma das atividades previstas para o primeiro biênio. Em  {\color{red}vermelho} encontra-se o mês de setembro.}
	\label{t1_cronograma}
\end{table}

\end{frame}

	
\begin{frame}
\begin{table}[H]
	\centering
	
	% definindo o tamanho da fonte para small
	% outros possíveis tamanhos: footnotesize, scriptsize
	\begin{footnotesize}
		
		% redefinindo o espaçamento das colunas
		\setlength{\tabcolsep}{1pt}
		
		% \cline é semelhante ao \hline, porém é possível indicar as colunas que terão essa a linha horizontal
		% \multicolumn{10}{c|}{Meses} indica que dez colunas serão mescladas e a palavra Meses estará centralizada dentro delas.
		\rotatebox{0}{
			\begin{tabular}{|c|c|c|c|c|c|c|c|c|c|c|c|c|c|c|c|c|c|c|c|c|c|c|c|c|}\hline
				& \multicolumn{24}{c|}{Meses}\\ \cline{2-25}
				\raisebox{1.5ex}{Etapa} & 25 & 26 & 27 & 28 & 29 & 30 & 31 & 32 & 33 & 34 & 35 & 36 & 37 & 38 & 39 & 40 & 41 & 42 & 43 & 44 & 45 & 46 & 47 & 48 \\ \hline
				
				Pesquisa na Literatura & X & X & X & X & X & X & & & & & & & & & & & & & & & & & & \\ \hline
				Disciplinas & & & & & & & & & & & & & & & & & & & & & & & & \\ \hline
				Formulação da Rede & & & & & & & & & & & & & & & & & & & & & & & & \\ \hline
				Treino & & & & & & & & & & & & & & & & & & & & & & & & \\ \hline
				Resultado & & X & X & X & X & X & X & X & & & & & & & & & & & & & & & & \\ \hline
				Artigo 1 & & X & X & X & & & & & & & & & & & & & & & & & & & & \\ \hline
				Artigo 2 & & & & & X & X & X & X & X & & & & & & & & & & & & & & & \\ \hline
				Tese & & & & & & & & & & & & & & & X & X & X & X & X & X & X & & & \\ \hline
				
			\end{tabular}
		}
	\end{footnotesize}
	\caption{Cronograma das atividades previstas para o segundo biênio.}
	\label{t2_cronograma}
\end{table}

\end{frame}	
	
	
\section{Bibliografia}
	\begin{frame}[allowframebreaks]{Bibliografia}
	%\frametitle{Bibliografia}
	\beamertemplatetextbibitems
	\tiny
	\bibliographystyle{apalike}
	\bibliography{references.bib}
	\end{frame}

\makeatother
{\nologo
\begin{frame}
%\titlepage
\begin{figure}
\includegraphics[scale=0.25]{Imagens/logonvertical.jpg}
\end{figure}
\begin{center}
\begin{minipage}{0.77\textwidth}
\small
\begin{center}
Rua General José Cristino, 77 CEP 20921-400\\
Rua General Bruce, 586 CEP 20921-030\\
Bairro Imperial de São Cristóvão, Rio de Janeiro - RJ\\
PABX: 55 21 3504-9100\\
\url{www.on.br}
\end{center}
\end{minipage}
\end{center}
\end{frame}
}
\end{document}
